\chapter{Análise Experimental}
\label{ch:analise-experimental}

\section{Método de Experimentação}
\label{sec:analise-experimental:metodo-de-experimentacao}

Os resultados foram obtidos a partir da utilização do módulo de instrumentação fornecido pelos professores.\ Este módulo permite medir o tempo de execução de uma função, bem como o número de instruções executadas.

Foi utilizado um computador com as seguintes especificações:

\begin{itemize}
    \item \textbf{CPU:} 11th Gen Intel i7-1165G7 (8) @ 4.700GHz;
    \item \textbf{GPU:} Intel TigerLake-LP GT2 [Iris Xe Graphics];
    \item \textbf{RAM:} 16GB;
    \item \textbf{Kernel:} 6.6.9-zen1-1-zen (Arch Linux x86\_64);
\end{itemize}

\section{Resultados}
\label{sec:analise-experimental:resultados}

Para a experimentação foram usados 5 grafos diferentes as seguintes dimensões:

\begin{itemize}
    \item Grafo 1--7 vértices e 9 arestas;
    \item Grafo 2--7 vértices e 12 arestas;
    \item Grafo 3--7 vértices e 11 arestas;
    \item Grafo 4--13 vértices e 15 arestas;
    \item Grafo 5--15 vértices e 25 arestas;
\end{itemize}

\begin{table}[H]
    \centering
    \label{tab:resultados}
    \begin{tabular}{|c|c|c|c|c|c|c|c|c|c|}

        \hline
        \multicolumn{2}{|c|}{\textbf{Grafo 1}} & \multicolumn{2}{|c|}{\textbf{Grafo 2}} & \multicolumn{2}{|c|}{\textbf{Grafo 3}} & \multicolumn{2}{|c|}{\textbf{Grafo 4}} & \multicolumn{2}{|c|}{\textbf{Grafo 5}} \\
        \hline
        \textbf{Versão} & \textbf{Caltime} & \textbf{Versão} & \textbf{Caltime} & \textbf{Versão} & \textbf{Caltime} & \textbf{Versão} & \textbf{Caltime} & \textbf{Versão} & \textbf{Caltime} \\
        \hline
        V1 & 0.010 ms & V1 & 0.006 ms & V1 & 0.007 ms & V1 & 0.011 ms & V1 & 0.014 ms\\
        \hline
        V2 & 0.001 ms & V2 & 0.002 ms & V2 & 0.002 ms & V2 & 0.002 ms & V2 & 0.003 ms\\
        \hline
        V3 & 0.001 ms & V3 & 0.002 ms & V3 & 0.001 ms & V3 & 0.002 ms & V3 & 0.002 ms\\
        \hline

    \end{tabular}
    \caption{Resultados obtidos}
\end{table}

\section{Análise dos Resultados}
\label{sec:analise-experimental:analise-dos-resultados}

Como se pode verificar pela tabela \ref{tab:resultados}, a versão 1 é a mais lenta, isto era esperado visto que esta versão precisa de criar uma cópia do grafo para chegar ao resultado.\ Por mais que as versões 2 e 3 sejam bastante semelhantes, a versão 3 é ligeiramente mais otimizada, o que se nota especialmente se escalarmos o número de vértices e arestas.