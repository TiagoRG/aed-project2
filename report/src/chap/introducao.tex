%! Author = TiagoRG
%! GitHub = https://github.com/TiagoRG

\chapter{Introdução}
\label{ch:introducao}
% Conteúdo da introdução aqui

\section{Contextualização}
\label{sec:introducao:contextualizacao}

Neste trabalho será desenvolvido um programa que trabalha com grafos.\ O programa conseguirá criar ficheiros, quer a partir de funções, quer a partir de um ficheiro de texto.\ Permitirá ainda realizar a ordenação topológica de um grafo recorrendo a um de três algoritmos implementados, sendo eles:

\begin{itemize}
    \item \ref{sec:analise-formal:algoritmo-1} através da cópia do grafo e remoção de vértices sem arestas de entrada;
    \item \ref{sec:analise-formal:algoritmo-2} utilizando uma lista auxiliar com os vértices marcados;
    \item \ref{sec:analise-formal:algoritmo-3} recorrendo a uma fila de prioridade (first in, first out).
\end{itemize}

Para cada um destes algoritmos será efetuada a respetiva análise de complexidade.\ Será ainda feita uma análise comparativa entre os três algoritmos, para determinar qual o mais eficiente.

\section{Estruturas de Dados}
\label{sec:introducao:estruturas-de-dados}

Os grafos são representados com a seguinte estrutura:

\begin{itemize}
    \item int isDigraph - indica se o grafo é direcionado ou não;
    \item int isComplete - indica se o grafo é completo ou não;
    \item int isWeighted - indica se o grafo tem pesos nas arestas ou não;
    \item unsigned int numVertices - número de vértices do grafo;
    \item unsigned int numEdges - número de arestas do grafo;
    \item List *verticesList - lista de vértices do grafo;
\end{itemize}

\vspace*{5mm}

Os vértices são representados com a seguinte estrutura:

\begin{itemize}
    \item unsigned int id - identificador do vértice;
    \item unsigned int inDegree - grau de entrada do vértice;
    \item unsigned int outDegree - grau de saída do vértice;
    \item List *edgesList - lista de arestas do vértice;
\end{itemize}

\vspace*{5mm}

As arestas são representadas com a seguinte estrutura:

\begin{itemize}
    \item unsigned int adjVertex - identificador do vértice de destino;
    \item double weight - peso da aresta;
\end{itemize}

\vspace*{5mm}

As listas estão implementadas como listas ordenadas, sendo que a ordenação é feita por ordem crescente do identificador do vértice de destino da aresta.\ A lista está implementada no ficheiro \textit{SortedList.c} e \textit{SortedList.h}.